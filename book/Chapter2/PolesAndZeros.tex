\section{Poles and Zeros} % ---------------------------------------------

With transfer functions introduced, I want to spend a little more time describing poles and zeros. ``What do you mean `poles and zeros'?'' you may ask. What most people mean with poles and zeros are the polynomial roots of the denominator and numerator (respectively) of a transfer function.

\begin{equation}
G(s) = \frac{s+10}{s(s^2+5s+6)}
\label{pz_func_example}
\end{equation}

Take the transfer function represented by Equation \ref{pz_func_example}. There are several values of $s$ that result in the numerator or denominator equalling zero. For the numerator, when $s = -10$ the transfer function will equal zero (due to the numerator being zero). This value of -10, by no coincidence, is a ``zero'' of the transfer function $G(s)$.

Additionally, there are multiple values of $s$ that would result in the denominator being zero. Those values are 0, -2, and -3. Now, if we call the top roots ``zeros'' because the transfer function goes to zero, we \emph{don't} call the roots that cause it to go to infinity ``infinities'' or ``divides by zero.'' Instead, imagine a 3D surface, with the two ``flat'' axes being the real and imaginary axes, and the ``height'' being the magnitude. The general shape around one of the denominator roots kind of resembles a tent with a big center pole going upward to infinity (because the function is some number divided by zero). So the thing that causes the tent shape is a ``pole.'' I didn't make this up; just roll with it and call the denominator roots ``poles'' and everything will be fine.

On a side note, there's no reason we cannot make a Bode plot of the transfer function in Equation \ref{pz_func_example}. That's given below in Figure \ref{pz_ex_bode}.

\begin{figure}[h]
\centering
\includegraphics[scale=0.5]{pz_example_bode.png}
\caption{Bode Plot for TF in Equation \ref{pz_func_example}}\label{pz_ex_bode}
\end{figure}

Go back to Equation \ref{pz_func_example} and think about the numerator. As $s$ changes, how does the gain and phase of just that term change? When $s$ is small, the fixed ``10'' dominates the gain, with no phase shift. As $s$ increases, both the gain and the phase shift increase. A similar thought process can be done with the denominator. Remember that all the terms are "acting" on the TF's total gain and phase shift at all times.

Take the generic form of a pole or zero (pulled out of the transfer function and set up on its own):

\begin{equation}
\bigg(\frac{s}{\omega_0} + 1\bigg)
\end{equation}

The frequency $\omega_0$ is called the ``corner frequency,'' ``3dB frequency,'' or just ``frequency'' of the respective pole or zero. (This might not be a real number for a root with complex solutions.) There are two key points to keep in mind with respect to this frequency (mentioned in passing previously, but reiterated here because they're important):

\begin{itemize}
\item At the corner frequency, the gain of that specific pole/zero starts shifting by 20dB / decade. (At the corner frequency, it has already shifted by 3dB, but it starts majorly shifting here.)
\item At one tenth the corner frequency, the phase shift contribution of that specific pole/zero starts becoming noticeable. At the corner frequency, it has shifted $45^{\circ}$. At ten times the corner frequency, the phase shift has mostly stabilize at $90^{\circ}$.
\end{itemize}

The direction of the phase or gain shifts depend on the form of the pole or zero. Typically, poles cause a negative shift in phase and gain, and zeros cause a positive shift in phase and gain. (Looking backwards, Figure \ref{simple_LPF_bode} is the graph of a pole with a corner frequency of 3.18Hz.) There are a few exceptions to this, such as a ``right-hand-plane zero,'' but that'll be covered later.